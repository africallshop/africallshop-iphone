\section{Basic call}
\label{group__basic__call__tutorials}\index{Basic call@{Basic call}}
This program is a {\itshape very} simple usage example of liblinphone. It just takes a sip-\/uri as first argument and attempts to call it


\begin{DoxyCodeInclude}

\textcolor{comment}{/*}
\textcolor{comment}{linphone}
\textcolor{comment}{Copyright (C) 2010  Belledonne Communications SARL }
\textcolor{comment}{ (simon.morlat@linphone.org)}
\textcolor{comment}{}
\textcolor{comment}{This program is free software; you can redistribute it and/or}
\textcolor{comment}{modify it under the terms of the GNU General Public License}
\textcolor{comment}{as published by the Free Software Foundation; either version 2}
\textcolor{comment}{of the License, or (at your option) any later version.}
\textcolor{comment}{}
\textcolor{comment}{This program is distributed in the hope that it will be useful,}
\textcolor{comment}{but WITHOUT ANY WARRANTY; without even the implied warranty of}
\textcolor{comment}{MERCHANTABILITY or FITNESS FOR A PARTICULAR PURPOSE.  See the}
\textcolor{comment}{GNU General Public License for more details.}
\textcolor{comment}{}
\textcolor{comment}{You should have received a copy of the GNU General Public License}
\textcolor{comment}{along with this program; if not, write to the Free Software}
\textcolor{comment}{Foundation, Inc., 59 Temple Place - Suite 330, Boston, MA  02111-1307, USA.}
\textcolor{comment}{*/}

\textcolor{preprocessor}{#ifdef IN\_LINPHONE}
\textcolor{preprocessor}{}\textcolor{preprocessor}{#include "linphonecore.h"}
\textcolor{preprocessor}{#else}
\textcolor{preprocessor}{}\textcolor{preprocessor}{#include "linphone/linphonecore.h"}
\textcolor{preprocessor}{#endif}
\textcolor{preprocessor}{}
\textcolor{preprocessor}{#include <signal.h>}

\textcolor{keyword}{static} bool\_t running=TRUE;

\textcolor{keyword}{static} \textcolor{keywordtype}{void} stop(\textcolor{keywordtype}{int} signum)\{
        running=FALSE;
\}

\textcolor{comment}{/*}
\textcolor{comment}{ * Call state notification callback}
\textcolor{comment}{ */}
\textcolor{keyword}{static} \textcolor{keywordtype}{void} call\_state\_changed(LinphoneCore *lc, LinphoneCall *call, 
      LinphoneCallState cstate, \textcolor{keyword}{const} \textcolor{keywordtype}{char} *msg)\{
        \textcolor{keywordflow}{switch}(cstate)\{
                \textcolor{keywordflow}{case} LinphoneCallOutgoingRinging:
                        printf(\textcolor{stringliteral}{"It is now ringing remotely !\(\backslash\)n"});
                \textcolor{keywordflow}{break};
                \textcolor{keywordflow}{case} LinphoneCallOutgoingEarlyMedia:
                        printf(\textcolor{stringliteral}{"Receiving some early media\(\backslash\)n"});
                \textcolor{keywordflow}{break};
                \textcolor{keywordflow}{case} LinphoneCallConnected:
                        printf(\textcolor{stringliteral}{"We are connected !\(\backslash\)n"});
                \textcolor{keywordflow}{break};
                \textcolor{keywordflow}{case} LinphoneCallStreamsRunning:
                        printf(\textcolor{stringliteral}{"Media streams established !\(\backslash\)n"});
                \textcolor{keywordflow}{break};
                \textcolor{keywordflow}{case} LinphoneCallEnd:
                        printf(\textcolor{stringliteral}{"Call is terminated.\(\backslash\)n"});
                \textcolor{keywordflow}{break};
                \textcolor{keywordflow}{case} LinphoneCallError:
                        printf(\textcolor{stringliteral}{"Call failure !"});
                \textcolor{keywordflow}{break};
                \textcolor{keywordflow}{default}:
                        printf(\textcolor{stringliteral}{"Unhandled notification %i\(\backslash\)n"},cstate);
        \}
\}

\textcolor{keywordtype}{int} main(\textcolor{keywordtype}{int} argc, \textcolor{keywordtype}{char} *argv[])\{
        LinphoneCoreVTable vtable=\{0\};
        LinphoneCore *lc;
        LinphoneCall *call=NULL;
        \textcolor{keyword}{const} \textcolor{keywordtype}{char} *dest=NULL;

        \textcolor{comment}{/* take the destination sip uri from the command line arguments */}
        \textcolor{keywordflow}{if} (argc>1)\{
                dest=argv[1];
        \}

        signal(SIGINT,stop);

\textcolor{preprocessor}{#ifdef DEBUG}
\textcolor{preprocessor}{}        linphone_core_enable_logs(NULL); \textcolor{comment}{/*enable liblinphone logs.*/}
\textcolor{preprocessor}{#endif}
\textcolor{preprocessor}{}        \textcolor{comment}{/* }
\textcolor{comment}{         Fill the LinphoneCoreVTable with application callbacks.}
\textcolor{comment}{         All are optional. Here we only use the call\_state\_changed callbacks}
\textcolor{comment}{         in order to get notifications about the progress of the call.}
\textcolor{comment}{         */}
        vtable.call_state_changed=call\_state\_changed;

        \textcolor{comment}{/*}
\textcolor{comment}{         Instanciate a LinphoneCore object given the LinphoneCoreVTable}
\textcolor{comment}{        */}
        lc=linphone_core_new(&vtable,NULL,NULL,NULL);

        \textcolor{keywordflow}{if} (dest)\{
                \textcolor{comment}{/*}
\textcolor{comment}{                 Place an outgoing call}
\textcolor{comment}{                */}
                call=linphone_core_invite(lc,dest);
                \textcolor{keywordflow}{if} (call==NULL)\{
                        printf(\textcolor{stringliteral}{"Could not place call to %s\(\backslash\)n"},dest);
                        \textcolor{keywordflow}{goto} end;
                \}\textcolor{keywordflow}{else} printf(\textcolor{stringliteral}{"Call to %s is in progress..."},dest);
                linphone_call_ref(call);
        \}
        \textcolor{comment}{/* main loop for receiving notifications and doing background linphonecore work: */}
        \textcolor{keywordflow}{while}(running)\{
                linphone_core_iterate(lc);
                ms\_usleep(50000);
        \}
        \textcolor{keywordflow}{if} (call && linphone_call_get_state(call)!=LinphoneCallEnd)\{
                \textcolor{comment}{/* terminate the call */}
                printf(\textcolor{stringliteral}{"Terminating the call...\(\backslash\)n"});
                linphone_core_terminate_call(lc,call);
                \textcolor{comment}{/*at this stage we don't need the call object */}
                linphone_call_unref(call);
        \}

end:
        printf(\textcolor{stringliteral}{"Shutting down...\(\backslash\)n"});
        linphone_core_destroy(lc);
        printf(\textcolor{stringliteral}{"Exited\(\backslash\)n"});
        \textcolor{keywordflow}{return} 0;
\}

\end{DoxyCodeInclude}
 