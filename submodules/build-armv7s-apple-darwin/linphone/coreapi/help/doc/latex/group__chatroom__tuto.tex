\section{Chat room and messaging}
\label{group__chatroom__tuto}\index{Chat room and messaging@{Chat room and messaging}}
This program is a {\itshape very} simple usage example of liblinphone, desmonstrating how to send/receive S\-I\-P M\-E\-S\-S\-A\-G\-E from a sip uri identity passed from the command line. \par
Argument must be like sip\-:{\tt jehan@sip.\-linphone.\-org} . \par
 ex chatroom sip\-:{\tt jehan@sip.\-linphone.\-org} \par
 
\begin{DoxyCodeInclude}

\textcolor{comment}{/*}
\textcolor{comment}{linphone}
\textcolor{comment}{Copyright (C) 2010  Belledonne Communications SARL }
\textcolor{comment}{}
\textcolor{comment}{This program is free software; you can redistribute it and/or}
\textcolor{comment}{modify it under the terms of the GNU General Public License}
\textcolor{comment}{as published by the Free Software Foundation; either version 2}
\textcolor{comment}{of the License, or (at your option) any later version.}
\textcolor{comment}{}
\textcolor{comment}{This program is distributed in the hope that it will be useful,}
\textcolor{comment}{but WITHOUT ANY WARRANTY; without even the implied warranty of}
\textcolor{comment}{MERCHANTABILITY or FITNESS FOR A PARTICULAR PURPOSE.  See the}
\textcolor{comment}{GNU General Public License for more details.}
\textcolor{comment}{}
\textcolor{comment}{You should have received a copy of the GNU General Public License}
\textcolor{comment}{along with this program; if not, write to the Free Software}
\textcolor{comment}{Foundation, Inc., 59 Temple Place - Suite 330, Boston, MA  02111-1307, USA.}
\textcolor{comment}{*/}

\textcolor{preprocessor}{#ifdef IN\_LINPHONE}
\textcolor{preprocessor}{}\textcolor{preprocessor}{#include "linphonecore.h"}
\textcolor{preprocessor}{#else}
\textcolor{preprocessor}{}\textcolor{preprocessor}{#include "linphone/linphonecore.h"}
\textcolor{preprocessor}{#endif}
\textcolor{preprocessor}{}
\textcolor{preprocessor}{#include <signal.h>}

\textcolor{keyword}{static} bool\_t running=TRUE;

\textcolor{keyword}{static} \textcolor{keywordtype}{void} stop(\textcolor{keywordtype}{int} signum)\{
        running=FALSE;
\}
\textcolor{keywordtype}{void} text\_received(LinphoneCore *lc, LinphoneChatRoom *room, \textcolor{keyword}{const} 
      LinphoneAddress *from, \textcolor{keyword}{const} \textcolor{keywordtype}{char} *message) \{
        printf(\textcolor{stringliteral}{" Message [%s] received from [%s] \(\backslash\)n"},message,
      linphone_address_as_string (from));
\}


LinphoneCore *lc;
\textcolor{keywordtype}{int} main(\textcolor{keywordtype}{int} argc, \textcolor{keywordtype}{char} *argv[])\{
        LinphoneCoreVTable vtable=\{0\};

        \textcolor{keywordtype}{char}* dest\_friend=NULL;


        \textcolor{comment}{/* takes   sip uri  identity from the command line arguments */}
        \textcolor{keywordflow}{if} (argc>1)\{
                dest\_friend=argv[1];
        \}

        signal(SIGINT,stop);
\textcolor{comment}{//#define DEBUG}
\textcolor{preprocessor}{#ifdef DEBUG}
\textcolor{preprocessor}{}        linphone_core_enable_logs(NULL); \textcolor{comment}{/*enable liblinphone logs.*/}
\textcolor{preprocessor}{#endif}
\textcolor{preprocessor}{}        \textcolor{comment}{/* }
\textcolor{comment}{         Fill the LinphoneCoreVTable with application callbacks.}
\textcolor{comment}{         All are optional. Here we only use the text\_received callback}
\textcolor{comment}{         in order to get notifications about incoming message.}
\textcolor{comment}{         */}
        vtable.text\_received=text\_received;

        \textcolor{comment}{/*}
\textcolor{comment}{         Instantiate a LinphoneCore object given the LinphoneCoreVTable}
\textcolor{comment}{        */}
        lc=linphone_core_new(&vtable,NULL,NULL,NULL);


        \textcolor{comment}{/*Next step is to create a chat root*/}
        LinphoneChatRoom* chat\_room = linphone_core_create_chat_room(lc,dest\_friend);

        linphone_chat_room_send_message(chat\_room,\textcolor{stringliteral}{"Hello world"}); \textcolor{comment}{/*sending message*/}

        \textcolor{comment}{/* main loop for receiving incoming messages and doing background linphone core work: */}
        \textcolor{keywordflow}{while}(running)\{
                linphone_core_iterate(lc);
                ms\_usleep(50000);
        \}

        printf(\textcolor{stringliteral}{"Shutting down...\(\backslash\)n"});
        linphone_chat_room_destroy(chat\_room);
        linphone_core_destroy(lc);
        printf(\textcolor{stringliteral}{"Exited\(\backslash\)n"});
        \textcolor{keywordflow}{return} 0;
\}

\end{DoxyCodeInclude}
 