The vpx multi-\/format codec S\-D\-K provides a unified interface amongst its supported codecs. This abstraction allows applications using this S\-D\-K to easily support multiple video formats with minimal code duplication or \char`\"{}special casing.\char`\"{} This section describes the interface common to all codecs. For codec-\/specific details, see the \hyperlink{group__codecs}{Supported Codecs} page.

The following sections are common to all codecs\-:
\begin{DoxyItemize}
\item \hyperlink{usage_usage_types}{Important Data Types}
\item \hyperlink{usage_usage_features}{Features}
\item \hyperlink{usage_usage_init}{Initialization}
\item \hyperlink{usage_usage_errors}{Error Handling}
\end{DoxyItemize}

Fore more information on decoder and encoder specific usage, see the following pages\-:
\begin{DoxyItemize}
\item \hyperlink{usage_decode}{Decoding}
\begin{DoxyItemize}
\item \hyperlink{usage_encode}{Encode}
\end{DoxyItemize}
\end{DoxyItemize}\hypertarget{usage_usage_types}{}\subsection{Important Data Types}\label{usage_usage_types}
There are two important data structures to consider in this interface.\hypertarget{usage_usage_ctxs}{}\subsubsection{Contexts}\label{usage_usage_ctxs}
A context is a storage area allocated by the calling application that the codec may write into to store details about a single instance of that codec. Most of the context is implementation specific, and thus opaque to the application. The context structure as seen by the application is of fixed size, and thus can be allocated with automatic storage or dynamically on the heap.

Most operations require an initialized codec context. Codec context instances are codec specific. That is, the codec to be used for the encoded video must be known at initialization time. See \hyperlink{group__codec_gad03e2dfa6ae511db7d25be6bbb336233}{vpx\-\_\-codec\-\_\-ctx\-\_\-t} for further information.\hypertarget{usage_usage_ifaces}{}\subsubsection{Interfaces}\label{usage_usage_ifaces}
A codec interface is an opaque structure that controls how function calls into the generic interface are dispatched to their codec-\/specific implementations. Applications \hyperlink{rfc2119_MUSTNOT}{M\-U\-S\-T N\-O\-T} attempt to examine or override this storage, as it contains internal implementation details likely to change from release to release.

Each supported codec will expose an interface structure to the application as an {\ttfamily extern} reference to a structure of the incomplete type \hyperlink{group__codec_gad654f3da60151f5dfef70aca00ef1e9e}{vpx\-\_\-codec\-\_\-iface\-\_\-t}.\hypertarget{usage_usage_features}{}\subsection{Features}\label{usage_usage_features}
Several \char`\"{}features\char`\"{} are defined that are optionally implemented by codec algorithms. Indeed, the same algorithm may support different features on different platforms. The purpose of defining these features is that when they are implemented, they conform to a common interface. The features, or capabilities, of an algorithm can be queried from it's interface by using the \hyperlink{group__codec_ga43adff58759093401235fb99247c82b8}{vpx\-\_\-codec\-\_\-get\-\_\-caps()} method. Attempts to invoke features not supported by an algorithm will generally result in \hyperlink{group__codec_ggada1084710837ad363b92f2379dd2b8d2a4470784ba5a3ef84dc0697d5489dd292}{V\-P\-X\-\_\-\-C\-O\-D\-E\-C\-\_\-\-I\-N\-C\-A\-P\-A\-B\-L\-E}.

Currently defined features available in both encoders and decoders include\-:
\begin{DoxyItemize}
\item \hyperlink{usage_xma}{External Memory Allocation}
\end{DoxyItemize}

Currently defined decoder features include\-:
\begin{DoxyItemize}
\item \hyperlink{usage_decode_usage_cb}{Callback Based Decoding}
\item \hyperlink{usage_decode_usage_postproc}{Postprocessing}
\end{DoxyItemize}\hypertarget{usage_usage_init}{}\subsection{Initialization}\label{usage_usage_init}
To initialize a codec instance, the address of the codec context and interface structures are passed to an initialization function. Depending on the \hyperlink{usage_usage_features}{Features} that the codec supports, the codec could be initialized in different modes. Most notably, the application may choose to use \hyperlink{usage_xma}{External Memory Allocation} mode to gain fine grained control over how and where memory is allocated for the codec.

To prevent cases of confusion where the A\-B\-I of the library changes, the A\-B\-I is versioned. The A\-B\-I version number must be passed at initialization time to ensure the application is using a header file that matches the library. The current A\-B\-I version number is stored in the preprocessor macros \hyperlink{group__codec_gaf7e9cad2df0f81679b881f46740ad097}{V\-P\-X\-\_\-\-C\-O\-D\-E\-C\-\_\-\-A\-B\-I\-\_\-\-V\-E\-R\-S\-I\-O\-N}, \hyperlink{group__encoder_gaa4f0b52293c08ba672429c3a03648b9d}{V\-P\-X\-\_\-\-E\-N\-C\-O\-D\-E\-R\-\_\-\-A\-B\-I\-\_\-\-V\-E\-R\-S\-I\-O\-N}, and \hyperlink{group__decoder_ga462b459e7ae13937e1eae1776245db12}{V\-P\-X\-\_\-\-D\-E\-C\-O\-D\-E\-R\-\_\-\-A\-B\-I\-\_\-\-V\-E\-R\-S\-I\-O\-N}. For convenience, each initialization function has a wrapper macro that inserts the correct version number. These macros are named like the initialization methods, but without the \-\_\-ver suffix.

The available initialization methods are\-:
\begin{DoxyItemize}
\item \hyperlink{group__encoder_ga3d490a2a9a6acd7c9ef82a603155f3cf}{vpx\-\_\-codec\-\_\-enc\-\_\-init} (calls \hyperlink{group__encoder_ga6ed21b96c481c0b6e1b543ef958a57a4}{vpx\-\_\-codec\-\_\-enc\-\_\-init\-\_\-ver()}) -\/ \hyperlink{group__decoder_ga8c2f0b12f1bd4927eb3c68b01eab19d3}{vpx\-\_\-codec\-\_\-dec\-\_\-init} (calls \hyperlink{group__decoder_ga26fe82cf8fd697f885935cea53be964f}{vpx\-\_\-codec\-\_\-dec\-\_\-init\-\_\-ver()})
\end{DoxyItemize}\hypertarget{usage_usage_errors}{}\subsection{Error Handling}\label{usage_usage_errors}
Almost all codec functions return an error status of type \hyperlink{group__codec_gada1084710837ad363b92f2379dd2b8d2}{vpx\-\_\-codec\-\_\-err\-\_\-t}. The semantics of how each error condition should be processed is clearly defined in the definitions of each enumerated value. Error values can be converted into A\-S\-C\-I\-I strings with the \hyperlink{group__codec_ga4d265df00d42b36a4f0e3eb83fc22c5e}{vpx\-\_\-codec\-\_\-error()} and \hyperlink{group__codec_gaaddf5c1f609ef18c7c8800d102fcefa6}{vpx\-\_\-codec\-\_\-err\-\_\-to\-\_\-string()} methods. The difference between these two methods is that \hyperlink{group__codec_ga4d265df00d42b36a4f0e3eb83fc22c5e}{vpx\-\_\-codec\-\_\-error()} returns the error state from an initialized context, whereas \hyperlink{group__codec_gaaddf5c1f609ef18c7c8800d102fcefa6}{vpx\-\_\-codec\-\_\-err\-\_\-to\-\_\-string()} can be used in cases where an error occurs outside any context. The enumerated value returned from the last call can be retrieved from the {\ttfamily err} member of the decoder context as well. Finally, more detailed error information may be able to be obtained by using the \hyperlink{group__codec_ga29273cb552ed1a437fe263c4a0a54300}{vpx\-\_\-codec\-\_\-error\-\_\-detail()} method. Not all errors produce detailed error information.

In addition to error information, the codec library's build configuration is available at runtime on some platforms. This information can be returned by calling \hyperlink{group__codec_ga20922bad85472e76d5f61c21cb423af7}{vpx\-\_\-codec\-\_\-build\-\_\-config()}, and is formatted as a base64 coded string (comprised of characters in the set \mbox{[}a-\/z\-\_\-a-\/\-Z0-\/9+/\mbox{]}). This information is not useful to an application at runtime, but may be of use to vpx for support.\hypertarget{usage_usage_deadline}{}\subsection{Deadline}\label{usage_usage_deadline}
Both the encoding and decoding functions have a {\ttfamily deadline} parameter. This parameter indicates the amount of time, in microseconds (us), that the application wants the codec to spend processing before returning. This is a soft deadline -- that is, the semantics of the requested operation take precedence over meeting the deadline. If, for example, an application sets a {\ttfamily deadline} of 1000us, and the frame takes 2000us to decode, the call to \hyperlink{group__decoder_ga3441e157a7a69108bca9a069f2ee8e0d}{vpx\-\_\-codec\-\_\-decode()} will return after 2000us. In this case the deadline is not met, but the semantics of the function are preserved. If, for the same frame, an application instead sets a {\ttfamily deadline} of 5000us, the decoder will see that it has 3000us remaining in its time slice when decoding completes. It could then choose to run a set of \hyperlink{usage_decode_usage_postproc}{Postprocessing} filters, and perhaps would return after 4000us (instead of the allocated 5000us). In this case the deadline is met, and the semantics of the call are preserved, as before.

The special value {\ttfamily 0} is reserved to represent an infinite deadline. In this case, the codec will perform as much processing as possible to yield the highest quality frame.

By convention, the value {\ttfamily 1} is used to mean \char`\"{}return as fast as
 possible.\char`\"{} \hypertarget{usage_decode}{}\subsection{Decoding}\label{usage_decode}
The \hyperlink{group__decoder_ga3441e157a7a69108bca9a069f2ee8e0d}{vpx\-\_\-codec\-\_\-decode()} function is at the core of the decode loop. It processes packets of compressed data passed by the application, producing decoded images. The decoder expects packets to comprise exactly one image frame of data. Packets \hyperlink{rfc2119_MUST}{M\-U\-S\-T} be passed in decode order. If the application wishes to associate some data with the frame, the {\ttfamily user\-\_\-priv} member may be set. The {\ttfamily deadline} parameter controls the amount of time in microseconds the decoder should spend working on the frame. This is typically used to support adaptive \hyperlink{usage_decode_usage_postproc}{Postprocessing} based on the amount of free C\-P\-U time. For more information on the {\ttfamily deadline} parameter, see \hyperlink{usage_usage_deadline}{Deadline}.

samples\hypertarget{usage_decode_usage_cb}{}\subsubsection{Callback Based Decoding}\label{usage_decode_usage_cb}
There are two methods for the application to access decoded frame data. Some codecs support asynchronous (callback-\/based) decoding \hyperlink{usage_usage_features}{Features} that allow the application to register a callback to be invoked by the decoder when decoded data becomes available. Decoders are not required to support this feature, however. Like all \hyperlink{usage_usage_features}{Features}, support can be determined by calling \hyperlink{group__codec_ga43adff58759093401235fb99247c82b8}{vpx\-\_\-codec\-\_\-get\-\_\-caps()}. Callbacks are available in both frame-\/based and slice-\/based variants. Frame based callbacks conform to the signature of \hyperlink{group__cap__put__frame_ga1c3d3d07ec4f907dc426bbd6d70862ec}{vpx\-\_\-codec\-\_\-put\-\_\-frame\-\_\-cb\-\_\-fn\-\_\-t} and are invoked once the entire frame has been decoded. Slice based callbacks conform to the signature of \hyperlink{group__cap__put__slice_ga344dbbf130aa9632aee94cee1f3cef44}{vpx\-\_\-codec\-\_\-put\-\_\-slice\-\_\-cb\-\_\-fn\-\_\-t} and are invoked after a subsection of the frame is decoded. For example, a slice callback could be issued for each macroblock row. However, the number and size of slices to return is implementation specific. Also, the image data passed in a slice callback is not necessarily in the same memory segment as the data will be when it is assembled into a full frame. For this reason, the application \hyperlink{rfc2119_MUST}{M\-U\-S\-T} examine the rectangles that describe what data is valid to access and what data has been updated in this call. For all their additional complexity, slice based decoding callbacks provide substantial speed gains to the overall application in some cases, due to improved cache behavior.\hypertarget{usage_decode_usage_frame_iter}{}\subsubsection{Frame Iterator Based Decoding}\label{usage_decode_usage_frame_iter}
If the codec does not support callback based decoding, or the application chooses not to make use of that feature, decoded frames are made available through the \hyperlink{group__decoder_ga0e231c3a5ce445fdb2268d741da97500}{vpx\-\_\-codec\-\_\-get\-\_\-frame()} iterator. The application initializes the iterator storage (of type \hyperlink{group__codec_ga6ea348f76b1f8a1fe50e14db684146c6}{vpx\-\_\-codec\-\_\-iter\-\_\-t}) to N\-U\-L\-L, then calls vpx\-\_\-codec\-\_\-get\-\_\-frame repeatedly until it returns N\-U\-L\-L, indicating that all images have been returned. This process may result in zero, one, or many frames that are ready for display, depending on the codec.\hypertarget{usage_decode_usage_postproc}{}\subsubsection{Postprocessing}\label{usage_decode_usage_postproc}
Postprocessing is a process that is applied after a frame is decoded to enhance the image's appearance by removing artifacts introduced in the compression process. It is not required to properly decode the frame, and is generally done only when there is enough spare C\-P\-U time to execute the required filters. Codecs may support a number of different postprocessing filters, and the available filters may differ from platform to platform. Embedded devices often do not have enough C\-P\-U to implement postprocessing in software. The filter selection is generally handled automatically by the codec, depending on the amount of time remaining before hitting the user-\/specified \hyperlink{usage_usage_deadline}{Deadline} after decoding the frame. \hypertarget{usage_encode}{}\subsection{Encode}\label{usage_encode}
The \hyperlink{group__encoder_gaf990542e2aeb389f05fae3e9c7803639}{vpx\-\_\-codec\-\_\-encode()} function is at the core of the encode loop. It processes raw images passed by the application, producing packets of compressed data. The {\ttfamily deadline} parameter controls the amount of time in microseconds the encoder should spend working on the frame. For more information on the {\ttfamily deadline} parameter, see \hyperlink{usage_usage_deadline}{Deadline}.

samples \hypertarget{usage_xma}{}\subsection{External Memory Allocation}\label{usage_xma}
Applications that wish to have fine grained control over how and where decoders allocate memory \hyperlink{rfc2119_MAY}{M\-A\-Y} make use of the e\-Xternal Memory Allocation (X\-M\-A) interface. Not all codecs support the X\-M\-A \hyperlink{usage_usage_features}{Features}.

To use a decoder in X\-M\-A mode, the decoder \hyperlink{rfc2119_MUST}{M\-U\-S\-T} be initialized with the vpx\-\_\-codec\-\_\-xma\-\_\-init\-\_\-ver() function. The amount of memory a decoder needs to allocate is heavily dependent on the size of the encoded video frames. The size of the video must be known before requesting the decoder's memory map. This stream information can be obtained with the \hyperlink{group__decoder_gadfee4664d644175d5aac1465ef11c4b0}{vpx\-\_\-codec\-\_\-peek\-\_\-stream\-\_\-info()} function, which does not require a constructed decoder context. If the exact stream is not known, a stream info structure can be created that reflects the maximum size that the decoder instance is required to support.

Once the decoder instance has been initialized and the stream information determined, the application calls the \hyperlink{group__cap__xma_ga802003c8ed203def368a361fe0e92f13}{vpx\-\_\-codec\-\_\-get\-\_\-mem\-\_\-map()} iterator repeatedly to get a list of the memory segments requested by the decoder. The iterator value should be initialized to N\-U\-L\-L to request the first element, and the function will return \hyperlink{group__codec_ggada1084710837ad363b92f2379dd2b8d2a452450a5adfcc14ef8a0ac12611dae21}{V\-P\-X\-\_\-\-C\-O\-D\-E\-C\-\_\-\-L\-I\-S\-T\-\_\-\-E\-N\-D} to signal the end of the list.

After each segment is identified, it must be passed to the codec through the \hyperlink{group__cap__xma_gac2df376a4d76282a5c117313182dcf53}{vpx\-\_\-codec\-\_\-set\-\_\-mem\-\_\-map()} function. Segments \hyperlink{rfc2119_MUST}{M\-U\-S\-T} be passed in the same order as they are returned from \hyperlink{group__cap__xma_ga802003c8ed203def368a361fe0e92f13}{vpx\-\_\-codec\-\_\-get\-\_\-mem\-\_\-map()}, but there is no requirement that \hyperlink{group__cap__xma_ga802003c8ed203def368a361fe0e92f13}{vpx\-\_\-codec\-\_\-get\-\_\-mem\-\_\-map()} must finish iterating before \hyperlink{group__cap__xma_gac2df376a4d76282a5c117313182dcf53}{vpx\-\_\-codec\-\_\-set\-\_\-mem\-\_\-map()} is called. For instance, some applications may choose to get a list of all requests, construct an optimal heap, and then set all maps at once with one call. Other applications may set one map at a time, allocating it immediately after it is returned from \hyperlink{group__cap__xma_ga802003c8ed203def368a361fe0e92f13}{vpx\-\_\-codec\-\_\-get\-\_\-mem\-\_\-map()}.

After all segments have been set using \hyperlink{group__cap__xma_gac2df376a4d76282a5c117313182dcf53}{vpx\-\_\-codec\-\_\-set\-\_\-mem\-\_\-map()}, the codec may be used as it would be in normal internal allocation mode.\hypertarget{usage_xma_usage_xma_seg_id}{}\subsubsection{Segment Identifiers}\label{usage_xma_usage_xma_seg_id}
Each requested segment is identified by an identifier unique to that decoder type. Some of these identifiers are private, while others are enumerated for application use. Identifiers not enumerated publicly are subject to change. Identifiers are non-\/consecutive.\hypertarget{usage_xma_usage_xma_seg_szalign}{}\subsubsection{Segment Size and Alignment}\label{usage_xma_usage_xma_seg_szalign}
The sz (size) and align (alignment) parameters describe the required size and alignment of the requested segment. Alignment will always be a power of two. Applications \hyperlink{rfc2119_MUST}{M\-U\-S\-T} honor the alignment requested. Failure to do so could result in program crashes or may incur a speed penalty.\hypertarget{usage_xma_usage_xma_seg_flags}{}\subsubsection{Segment Flags}\label{usage_xma_usage_xma_seg_flags}
The flags member of the segment structure indicates any requirements or desires of the codec for the particular segment. The \hyperlink{vpx__codec_8h_a0a5b645a72e936464b1a9144bcbabed2}{V\-P\-X\-\_\-\-C\-O\-D\-E\-C\-\_\-\-M\-E\-M\-\_\-\-Z\-E\-R\-O} flag indicates that the segment \hyperlink{rfc2119_MUST}{M\-U\-S\-T} be zeroed by the application prior to passing it to the application. The \hyperlink{vpx__codec_8h_a7165be67097e7553d0983962a8828cf5}{V\-P\-X\-\_\-\-C\-O\-D\-E\-C\-\_\-\-M\-E\-M\-\_\-\-W\-R\-O\-N\-L\-Y} flag indicates that the segment will only be written into by the decoder, not read. If this flag is not set, the application \hyperlink{rfc2119_MUST}{M\-U\-S\-T} insure that the memory segment is readable. On some platforms, framebuffer memory is writable but not readable, for example. The \hyperlink{vpx__codec_8h_a2d57a8f37a8f8c9cc3497759ffc52b2a}{V\-P\-X\-\_\-\-C\-O\-D\-E\-C\-\_\-\-M\-E\-M\-\_\-\-F\-A\-S\-T} flag indicates that the segment will be frequently accessed, and that it should be placed into fast memory, if any is available. The application \hyperlink{rfc2119_MAY}{M\-A\-Y} choose to place other segments in fast memory as well, but the most critical segments will be identified by this flag.\hypertarget{usage_xma_usage_xma_seg_basedtor}{}\subsubsection{Segment Base Address and Destructor}\label{usage_xma_usage_xma_seg_basedtor}
For each requested memory segment, the application must determine the address of a memory segment that meets the requirements of the codec. This address is set in the {\ttfamily base} member of the \hyperlink{structvpx__codec__mmap}{vpx\-\_\-codec\-\_\-mmap} structure. If the application requires processing when the segment is no longer used by the codec (for instance to deallocate it or close an associated file descriptor) the {\ttfamily dtor} and {\ttfamily priv} members can be set. 